\definepapersize[A4]
\setupbodyfont[14pt]
\noheaderandfooterlines 
\starttext
1. In Ricart and Agrawala's distributed mutual exclusion algorithm, show that: \blank
(a) Processes enter their critical sections in the order of their request timestamps.

\blank[2*big]

Suppose here process {\bf j} enters critical section before process {\bf i}, and ${\bf t_j > t_i}$. \\
Before process {\bf j} enters critical section, according to {\bf guard 4}, N in process {\bf j} should equal to n - 1, but {\bf guard 2a} guarantees that process {\bf i} can not send an acknowledgment to process {\bf j} due to $t_j > t_i$.\\
So, N in process {\bf j} must be less than n - 1, which means process {\bf j} could not enter critical section before process {\bf i}.

\blank[2*big]

(b) Correctness is guaranteed even if the channels are not FIFO.

\blank[2*big]

Because this algorithm maintains a acyclic waitfor chain. Each process waits for the other process ahead of it to send an acknowledgment. \\
Still use the above example, even if process {\bf i} received m(j, req, $t_j$) earlier than process {\bf j} received m(i, req, $t_i$), and ${\bf t_j > t_i}$, and due to the algorithm, process {\bf i} has a higher rank than process {\bf j} in waitfor chain, therefore, process {\bf i} enters the critical section before process {\bf j}. So, it is unrelated to channel, but the timestamp.

\stoptext